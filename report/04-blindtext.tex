%!TEX root = ../dhbw-report.tex

\Blinddocument
\chapter{Pictures}

	\section{Example for a basic figure}

	\Blindtext[1]
	\Blindtext[1]

	\section{Examplew for subfigures}
	\Blindtext[1]
	\begin{figure}[htb] % requires "subfigure" package
		\subfigure[DHBW]{\includegraphics[width=0.49\textwidth]{./pic/logo/dhbw.png}}
		\caption{Logos nex to each other}
		\label{fig:logos1}
	\end{figure}

	\Blindtext[1]

	\begin{figure}[htb] % requires "subfigure" package
		\subfigure[DHBW]{\includegraphics[width=0.49\textwidth]{./pic/logo/dhbw.png}}
		\caption{Logos underneath each other}
		\label{fig:logos2}
	\end{figure}

	\Blindtext[1]

\chapter{Tables}

	\section{Example for a basic table (tabular)}
	Look at table~\ref{tab:tabular_example} to see a nice example for a table.
	\Blindtext[1]
	\begin{table}[ht]
		\begin{tabular}{|l|c|r|}% number of columns with or without border
			\hline
			Options & Description & Something \\ \hline
			l & left orientated & each column\\ \hline
			c & center alignment & is as wide as its \\ \hline
			r & guess what & largest cell needs to be, no wraping of text is done.\\ \hline
		\end{tabular}
		\caption{some usefull text}
		\label{tab:tabular_example}
	\end{table}
	\Blindtext[1]

	\section[A table with wrapping lines]{Example for a table with wrapping lines (tabularx)}
	Look at table~\ref{tab:tabularx_example} to see a nice example for a table.
	\Blindtext[1]
	\begin{table}[hb]
		\centering
		\begin{tabularx}{0.9\textwidth}{|l|C|r|X|}% number of columns with or without border
			\hline
			Options & Description & Something & Something else\\\hline\hline
			l & left orientated & each traditional column & Each of the new column types \\
			C & center alignment, dynamic size & is as wide as its & is as wide as it is allowed to be.\\
			r & guess what & largest cell needs to be. & It is restricted by\\
			X & left orientation, dynamic size & - & the remaining size of the table not used by other columns, thus text wrap does happen \\
			\hline
		\end{tabularx}
		\caption{some usefull text}
		\label{tab:tabularx_example}
	\end{table}
	\Blindtext[1]

\chapter{Quotes}
	\blindtext
	\begin{quote}
		``Ein Plagiat (von lat. plagium, "`Menschenraub"', "`Raub der Seele"'[1]) ist die Vorlage fremden geistigen Eigentums bzw. eines fremden 
		Werkes als eigenes Werk oder als Teil eines eigenen Werkes. Dies kann sich auf eine wortwörtliche Übernahme, eine Bearbeitung, oder auch 
		die Darstellung von Ideen oder Argumenten beziehen.''\cite{wiki:plag} \ac{HTTP}
	\end{quote}
	\blindtext
